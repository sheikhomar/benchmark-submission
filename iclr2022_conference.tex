
\documentclass{article} % For LaTeX2e
\usepackage{iclr2022_conference,times}

% Optional math commands from https://github.com/goodfeli/dlbook_notation.
\input{math_commands.tex}

\usepackage{hyperref}
\usepackage{url}
\usepackage[capitalize]{cleveref}

\usepackage{hyperref}

\usepackage{amsthm}
\usepackage{amsfonts}           % \mathbb
\usepackage{amssymb}             % proof
\usepackage{xspace}             % \xspace
\usepackage{graphicx}
\usepackage{adjustbox}          % \adjustbox 
\usepackage{multicol}            
\usepackage{url}
\usepackage{enumerate, enumitem}  %enumerate environment with optional argument
\usepackage{subcaption}
% \usepackage[usenames]{xcolor} % for coordinating edits
\newcommand{\sout}[1]{\st{#1}}
\usepackage{hyperref}
\usepackage{amsmath, hyperref, nicefrac}
\usepackage[capitalize]{cleveref}
\usepackage{thm-restate}
\usepackage{parskip}
\usepackage{mathtools}
\usepackage{algorithm}
\usepackage{algorithmic}
\usepackage{dsfont}
\newcommand\dsone{\mathds{1}}




\newtheorem{lemma}{Lemma}[section]
\newtheorem{fact}{Fact}[section]
\newtheorem{corollary}{Corollary}[section]
\newtheorem{axiom}{Axiom}[section]
\newtheorem{cond}{Condition}[section]
\newtheorem{property}{Property}[section]
\newtheorem{proposition}{Proposition}[section]

\newtheorem{Conjecture}{Conjecture}[section]
%\newtheorem{Corollary}[Theorem]{Corollary}
\newtheorem{Definition}{Definition}[section]
\newtheorem{Lemma}{Lemma}[section]
\newtheorem{Remark}{Remark}[section]





\title{An Empirical Evaluation of $k$-Means Coresets}

% Authors must not appear in the submitted version. They should be hidden
% as long as the \iclrfinalcopy macro remains commented out below.
% Non-anonymous submissions will be rejected without review.

\author{Antiquus S.~Hippocampus, Natalia Cerebro \& Amelie P. Amygdale \thanks{ Use footnote for providing further information
about author (webpage, alternative address)---\emph{not} for acknowledging
funding agencies.  Funding acknowledgements go at the end of the paper.} \\
Department of Computer Science\\
Cranberry-Lemon University\\
Pittsburgh, PA 15213, USA \\
\texttt{\{hippo,brain,jen\}@cs.cranberry-lemon.edu} \\
\And
Ji Q. Ren \& Yevgeny LeNet \\
Department of Computational Neuroscience \\
University of the Witwatersrand \\
Joburg, South Africa \\
\texttt{\{robot,net\}@wits.ac.za} \\
\AND
Coauthor \\
Affiliation \\
Address \\
\texttt{email}
}

% The \author macro works with any number of authors. There are two commands
% used to separate the names and addresses of multiple authors: \And and \AND.
%
% Using \And between authors leaves it to \LaTeX{} to determine where to break
% the lines. Using \AND forces a linebreak at that point. So, if \LaTeX{}
% puts 3 of 4 authors names on the first line, and the last on the second
% line, try using \AND instead of \And before the third author name.


\colorlet{darkgreen}{green!45!black}

% \newcommand{\R}{\mathbb{R}}
\newcommand{\set}[1]{\{#1\}}
\newcommand{\etal}{et al.\xspace}
\newcommand{\dist}{\text{dist}}
% \newcommand{\eps}{\varepsilon}
\newcommand{\opt}{\text{OPT}}
\newcommand{\cost}{\text{cost}}
\newcommand{\calS}{\mathcal{S}}
\newcommand{\calP}{\mathcal{P}}
\newcommand{\calL}{\mathcal{L}}
\newcommand{\calR}{\mathcal{R}}
\newcommand{\calA}{\mathcal{A}}
\newcommand{\calT}{\mathcal{T}}
\newcommand{\ba}{\mathbf{a}}
\newcommand{\bb}{\mathbf{b}}
\newcommand{\bc}{\mathbf{c}}
\newcommand{\bd}{\mathbf{d}}
\newcommand{\bvf}{\mathbf{f}}
\newcommand{\bp}{\mathbf{p}}

\newcommand{\by}{\mathbf{p}}

% \newcommand{\E}{\mathbb{E}}
\newcommand{\pr}{\mathbb{P}}
\newcommand{\calE}{\mathcal{E}}
\newcommand{\calI}{\mathcal{I}}
\newcommand{\calF}{\mathcal{F}}
\newcommand{\calX}{\mathcal{X}}
\newcommand{\calB}{\mathcal{B}}
\newcommand{\calC}{\mathcal{C}}
\newcommand{\bI}{\bar{I}_i}
\newcommand{\cand}{\mathbb{C}}
\newcommand{\greedy}{\mathfrak{c}}
\newcommand{\A}{\mathcal{A}}
\newcommand{\poly}{\text{poly}}
\newcommand{\alg}{\greedy}
\newcommand{\centers}{\mathcal{C}}
\newcommand{\coreset}{\Omega}
\newcommand{\offset}{F}
\newcommand{\weight}{f}
\newcommand{\inner}{R_I}
\newcommand{\out}{R_O}
\newcommand{\main}{R_M}
\newcommand{\size}{\Gamma}
\newcommand{\polylog}{\text{polylog}}

\newcommand{\valuedelta}{\frac{\log^2 1/\eps}{2^{O(z\log z)}\min(\eps^2, \eps^z)}\left(k \log |\cand| + \log \log (1/\eps) + \log(1/\pi)\right)}

\newcommand\chris[1]{\textcolor{blue}{Chris: #1}}
\newcommand\omar[1]{\textcolor{green!60!black}{Omar: #1}}

\newcommand{\fix}{\marginpar{FIX}}
\newcommand{\new}{\marginpar{NEW}}



%\iclrfinalcopy % Uncomment for camera-ready version, but NOT for submission.
\begin{document}


\maketitle

\begin{abstract}
Coresets are among the most popular paradigms for summarizing data. In particular, there exist many high performance coresets for clustering problems such as $k$-means in both theory and practice. Curiously, there exists little work on comparing the quality of available $k$-means coresets. 

In this paper we perform such an evaluation. First, we show that it is computationally hard to compare the quality of not only two different coreset algorithms, but also of two different output of a (randomized) coreset algorithm.
In order to perform an empirical evaluation, we therefore have to work with heuristics. 
To this end, we propose and analyse a benchmark for coreset comparison. Using this benchmark and real-world data sets, we conduct an exhaustive evaluation of the most commonly used coreset implementations.
\end{abstract}


\section{Introduction}

The design and analysis of scalable algorithms has become an important research area over the past two decades. This is particularly important in data analysis, where even polynomial running time might not be enough to handle proverbial \emph{big data} sets.
One of the main approaches to deal with the scalability issue is to compress or sketch large data sets into smaller, more manageable ones. The aim of such compression methods is to preserve the properties of the original data, up to some small error, while significantly reducing the number of data points.

Among the most popular and successful paradigms in this line of research are \emph{coresets}. Informally, given a data set $A$, a coreset $S\subset A$ with respect to a given set of queries $Q$ and query function $f: A\times Q \rightarrow \mathbb{R}_{\geq 0}$ approximates the behaviour of $A$ for all queries up to some multiplicative distortion $D$ via
$$ \sup_{q\in Q} \max\left( \frac{f(S,q)}{f(A,q)},\frac{f(A,q)}{f(S,q)}\right) \leq D.$$
Coresets have been applied to a number of problems such as computational geometry \cite{AHV05,Chan09}, linear algebra \cite{IndykMGR20,maalouf2019fast}, and machine learning \cite{MRM21,MunteanuSSW18}. But the by far most intensively studied and arguably most successful applications of the coreset framework is the $k$-clustering problem.

Here we are given $n$ points $A$ with (potential unit) weights $w:A\rightarrow \mathbb{R}_{\geq 0}$ in some metric space with distance function $\dist$ and aim to find $k$ centers $C$ such that $$\cost_A(C):= \frac{1}{n} \sum_{p\in A}  \min_{c\in C} w(p)\cdot \dist^z(p,c)$$
\omar{Shouldn't the cost be:
$$\cost_A(C):= \sum_{p\in A} w(p)\cdot  \min_{c\in C} \dist^z(p,c)$$
}
is minimized. The most popular variant of this problem is probably the $k$-means problem in $d$-dimensional Euclidean space where $z=2$ and $\dist(x,y) = \sqrt{\sum_{i=1}^d (x_i-y_i)^2}$.



A $(k,\varepsilon)$-coreset is now a subset $\Omega\subset A$ with weights $w:\Omega\rightarrow \mathbb{R}_{\geq 0}$ such that for any set of $k$ centers $C$
\begin{equation}
\label{eq:coreset}
\sup_{C} \max\left( \frac{\cost_A(C)}{\cost_{\Omega}(C)},\frac{\cost_{\Omega}(C)}{\cost_{A}(C)}\right) \leq 1+\varepsilon.
\end{equation}
The coreset definition in \cref{eq:coreset} provides an upper bound for the distortion of all candidate solutions i.e., all possible $k$ clusterings. 
A \emph{weak coreset} is a relaxed guarantee that holds for optimal or nearly optimal clusterings of $A$ instead of all clusterings.

In a long line of work spanning the last 20 years\cite{BecchettiBC0S19,BravermanJKW21,Chen09,FL11,FeldmanSS20,
HaM04,HaK07,huang2020coresets,BravermanJKW21,LS10,SohlerW18}, the size of coresets has been steadily improved with the current state of the art yielding a coreset with $\tilde{O}(k\varepsilon^{-4})$ points for a distortion $D\leq (1+\varepsilon)$ due to Cohen-Addad, Saulpic, and Schwiegelshohn \cite{Cohen-AddadSS21}\footnote{We use $\tilde O(x)$ to hide $\log^c x$ terms for any constant $c$.}.

While we have a good grasp of the theoretical guarantees of these algorithms, our understanding of the empirical performance is somewhat lacking. Verifying coresets is important in practice. However, it is not trivial to assess the quality of a coreset. To accurately evaluate a given coreset, we would need to come up with a $k$ clustering $C$ which results in a maximal distortion. Our conjecture is that finding such a worst-case solution in the general case is co-NP hard, while we know it is co-NP hard for weak coresets.

Due to this difficulty, a common heuristic for evaluating coresets is as follows~\cite{AckermannMRSLS12,FGSSS13}. First, compute a coreset $\Omega$ with the available algorithm(s) using some input data $A$. Then, run an optimization algorithm on $\Omega$ to compute a $k$ clustering $C$. For randomized algorithms, the two steps are repeated multiple times. The \emph{best} coreset algorithm is considered to be the one which yields a clustering with the smallest cost.

The drawback of this evaluation method is that it mixes up the two separate tasks of coreset construction and optimization.
A coreset algorithm may yield a good clustering (with small cost) yet fail to produce a high quality coreset (with small distortion).
Consequently, the method is more likely to measure the performance of the underlying optimization problem, rather than evaluating coresets.


The purpose of this study is to systematically evaluate the quality of various coreset algorithms for $k$-means. As such, we develop a new evaluation procedure which estimates the distortion of coreset algorithms. On real-world data sets, we observe that while the evaluated coreset algorithms are generally able to find solutions with comparable costs, there is a stark difference in their distortions. This shows that differences between optimization and compression are readily observable in practice.

As a complement to our evaluation procedure on real-world data sets, we propose a benchmark framework for generating synthetic data sets. We argue why this benchmark has properties that results in hard instances for all known coreset constructions. We also show how to efficiently estimate the distortion of a candidate coreset on the benchmark.




\input{preliminaries}

\section{Coreset Algorithms}
\label{sec:algorithms}

Though the algorithms vary in details, coreset constructions come in one of the following two flavours:
\begin{enumerate}
\item Movement-based constructions: Such algorithms compute a clustering with $T$ points such that $\cost_A(T)\ll \opt$, where $\opt$ is the cost of an optimal $k$-means clustering. 
% \item Movement-based constructions: Such algorithms compute a coreset $\Omega$ with $T$ points given some input point set $A$ such that $\cost_{\Omega}(C)\ll \opt$, where $\opt$ is the cost of an optimal $k$-means clustering of $A$. 
The coreset guarantee then follows as a consequence of the triangle inequality. These algorithms all have an exponential dependency on the dimension $d$, and therefore have been overtaken by sampling-based methods. Nevertheless, these constructions are more robust to various constrained clustering formulations~\cite{HuangJV19,SSS19} and continue to be popular. Examples from theory include~\cite{FrahlS2005,HaM04}. 
\item Importance sampling: Points are sampled proportionate to their sensitivity which for a point $p$ is defined as $sens(p):=\sup_{C} \frac{\min_{c\in C} \dist^2(p,c)}{\cost_A(C)}$ and weighted by their inverse sampling probability. In terms of theoretical performance, sensitivity sampling has largely replaced movement based constructions, see for example~\cite{FeldmanL11,LangbergS10}.  
\end{enumerate}

Of course, there exist algorithms that draw on techniques from both, see for example~\cite{Cohen-AddadSS21}. In what follows, we will survey implementations of various coreset constructions that we will evaluate later.


\begin{description}
\item[StreamKM++~\cite{AckermannMRSLS12}] The popular $k$-means++ algorithm~\cite{ArV07} computes a set of centers $K$ by iteratively sampling a point $p$ in $A$ proportionate to $\min_{q\in K} \dist^2(p,q)$ and adding it to $K$. The procedure terminates once the desired number of centers has been reached. The first center is typically picked uniformly at random.
The StreamKM++ paper runs the $k$-means++ algorithms for $T$ iterations, where $T$ is the desired coreset size. At the end, every point $q$ in $K$ is weighted by the number of points in $A$ closest to it. While the construction has elements of important sampling, the analysis is largely movement-based. The provable bound required for the algorithm to compute a coreset is $O\left(\frac{k\log n}{\delta^{d/2}\varepsilon^d}\cdot \log^{d/2} \frac{k\log n}{\delta^{d/2}\varepsilon^d}\right)$.
\item[BICO~\cite{FGSSS13}] Combines the very fast, but poor quality clustering algorithm BIRCH~\cite{ZRL97} with the movement-based analysis from~\cite{FrahlS2005,HaM04}. The clustering is organized by way of a hierarchical decomposition: When adding a point $p$ to one of the coreset points $\Omega$ at level $i$, it first finds the closest point $q$ in $\Omega$. If $p$ is too far away from $q$, a new cluster is opened with center at $p$. Otherwise $p$ is either added to the same cluster as $q$, or, if adding $p$ to $q$'s cluster increases the clustering cost beyond a certain threshold, the algorithm attempts to add $p$ to the child-clusters of $q$. The procedure then continues recursively. The provable bound required for the algorithm to compute a coreset is $O\left(k\varepsilon^{-d-2}\log n\right)$.
\item[Sensitivity Sampling~\cite{FL11}] The simplest implementation of sensitivity sampling first computes an $(O(1),O(1))$ bicriteria approximation\footnote{An $(\alpha,\beta)$ bicriteria approximation computes an $\alpha$ approximation using $\beta\cdot k$ many centers.}, for example by running $k$-means++ for $2k$ iterations~\cite{Wei16}. Let $K$ be the $2k$ clustering thus computed and let $K_i$ be an arbitrary cluster of $K$ with center $q_i$. Subsequently, the algorithm picks points proportionate to $\frac{\dist^2(p,q)}{\cost_{K_i}(\{q_i\})} + \frac{1}{|K_i|}$ and weighs any point by its inverse sampling probability. Let $|\hat{K_i}|$ be the estimated number of points in the sample. Finally, the algorithm weighs each $q_i$ by $(1+\eps)\cdot |K_i| - |\hat{K_i}|$. The provable bound required for the algorithm to compute a coreset is $\tilde O\left(kd\varepsilon^{-4}\right)$ (\cite{FL11}),
$\tilde O\left(k\varepsilon^{-6}\right)$ (\cite{huang2020coresets}), or $\tilde O\left(k^2\varepsilon^{-4}\right)$ (\cite{BravermanJKW21}).
\item[Group Sampling~\cite{Cohen-AddadSS21}] First, the algorithm computes an $O(1)$ approximation (or a bicriteria approximation) $K$. Subsequently, the algorithm preprocesses the input into groups such that (1) for any two points $p,p'\in K_i$, their cost is identical up to constant factors and (2) for any two clusters $K_i,K_j$, their cost is identical up to constant factors. In every group, Group Sampling now samples points proportionate to their cost. The authors of~\cite{Cohen-AddadSS21} show that there always exist a partitioning into $\log^2 1/\varepsilon$ groups. Points not contained in a group are snapped to their closest center $q$ in $K$. $q$ is weighted by the number of points snapped to it. The provable bound required for the algorithm to compute a coreset is $\tilde O\left(k\varepsilon^{-4}\right)$ (\cite{Cohen-AddadSS21}).
\item[Ray Maker~\cite{harpeled2007raymaker}] The algorithm computes an initial solution with $k$ centers which is a constant factor approximation of the optimal clustering. Around each center, $O(1/\epsilon^{d-1})$ random rays are created which span the hyperplane. Next, each point $p \in A$ is snapped to its closest ray resulting in a set of one-dimensional points associated with each ray. Afterwards, a coreset is created for each ray by computing an optimal 1D clustering with $k^2/\epsilon^2$ centers and weighing each center by the number of points in each cluster. The final coreset is composed of the coresets computed for all the rays.
The provable bound required for the algorithm to compute a coreset is $O(k^3 \cdot \varepsilon^{-d-1})$. The algorithm has recently received some attention due to its applicability to the fair clustering problem~\cite{HuangJV19}.
\end{description}

\subsection{Dimension Reduction}
\label{sec:dim_reduction}

Finally, we also combine coreset constructions with a variety of dimension reduction techniques. Since the seminal paper by Feldman, Schmidt, and Sohler~\cite{FSS13}, most coreset algorithms have used some form of dimension reduction to eliminate the dependency on $d$, either by explicitly computing a low-dimensional embedding, see for example~\cite{FSS13,SoW18}, or by using the existence of a suitable embedding in the analysis~\cite{Cohen-AddadSS21,huang2020coresets}.

In particular, movement-based coresets often have an exponential dependency on the dimension, which can be alleviated with some form of dimension reduction, both in theory~\cite{SSS19} and in practice~\cite{KappmeierS015}.
Here the are two main techniques.

\begin{description}
\item[Principal Component Analysis:] Feldman, Schmidt, and Sohler~\cite{FSS13} showed that projecting an input $A$ onto the first $O(k/\varepsilon^2)$ principal components is a coreset, albeit in low dimension. The analysis was subsequently tightened by~\cite{CEMMP15} and extended to other center based cost functions by~\cite{SohlerW18}. Although its target dimension is generally worse than those based on random projections and terminal embeddings, there is nevertheless reasons for using PCA regardless: It removes noise and thus may make it easier to compute a high quality coreset.
\item[Terminal Embeddings:] Given a set of points $A$ in $\mathbb{R}^D$, a terminal embedding $f:\mathbb{R}^D\rightarrow \mathbb{R}^d$ preserves the pairwise distance between any point $p\in A$ and any point $q\in \mathbb{R}^D$ up to a $(1\pm \varepsilon)$ factor. The statement is related to the famous Johnson-Lindenstrauss lemma but it is stronger as it does not apply to only the pairwise distances of $A$. Nevertheless, the same target dimension is sufficient. Terminal embeddings were studied by~\cite{ElkinFN17,MahabadiMMR18,NaN18}, with Narayanan and Nelson \cite{NaN18} achieving an optimal target dimension of $O(\varepsilon^{-2}\log n)$, where $n$ is the number of points. For applications to coresets, we refer to \cite{BecchettiBC0S19,Cohen-AddadSS21,huang2020coresets}.
\end{description}

For an overview on practical aspects of dimension reduction, we refer to Venkatsubramanian and Wang~\cite{VenkatasubramanianW11}. In our evaluation, we  focus on using PCA. The performance of the algorithms when applying a Johnson-Lindenstrauss transformation does not affect the behaviour of an algorithm that only depends on pairwise distances. We note that terminal embeddings, combined with an iterative application of the coreset construction from \cite{BravermanJKW21}, can reduce the target dimension to a factor $\tilde{O}(\varepsilon^{-2} \log k)$. This is mainly of theoretical interest, as in practice the deciding factor wrt the target dimension is the precision, rather than dependencies on $\log n$ and $\log k$.

\section{Hardness of Coreset Evaluation and a Benchmark}

In this section, we first show that it is in general co-NP hard to evaluate the coreset distortion, given two point sets $A$ and $B$. Thereafter we describe the benchmark and it's properties.


\begin{proposition}
Given two point sets $A$ and $B$ in $\mathbb{R}^d$ and a sufficiently small (constant) $\varepsilon>0$, it is co-NP hard to decide whether $A$ is a $(k,\varepsilon)$-coreset of $B$.
\end{proposition}
\begin{proof}
First, we recall that for some $\varepsilon_0$ and candidate clustering cost $V$, it is NP-hard to decide whether there exists a clustering with cost in $[V,(1+\varepsilon_0)\cdot V]$.
Therefore, it is co-NP-hard to decide whether there exists a set of centers $C$ such that $\cost_A(C) \geq (1+\varepsilon_0)\cdot \cost_B(C)$.
\end{proof}

We remark that the possible values for $\varepsilon_0$ are determined by the current APX-hardness results. Assuming NP$\neq$P, $\varepsilon_0\approx 1.07$ and assuming UCG, $\varepsilon_0 \approx 1.17$~\cite{Cohen-AddadSL21,Cohen-AddadS19} for $k$-means in Euclidean spaces.

\subsection{Benchmark Construction}

In this section, we describe our benchmark. The benchmark has a parameters $\alpha$ which controls the number of points and dimensions.
For a given value of $k$ the benchmark consists of $n=k^\alpha$ points and $d=\alpha \cdot k$ rows.
It is recursively constructed as follows.

Denote by $\mathds{1}_k$ the $k$-dimensional all $1$ vector and by $v_i$ the $k$ dimensional vector with entries $(v_i)_j = \begin{cases}-\frac{1}{k} & \text{if } i\neq j\\
\frac{k-1}{k} & \text{if } i= j\end{cases}$.
For $\ell\leq \alpha$, recursively define the $k^{\ell}$ dimensional vector $v_i^{\ell} = \begin{bmatrix}
(v_i^{\ell-1})_1 \cdot \mathds{1}_k \\
(v_i^{\ell-1})_2 \cdot \mathds{1}_k \\
\vdots \\
(v_i^{\ell-1})_1 \cdot \mathds{1}_k
\end{bmatrix}$. Finally, set the column $t = a\cdot k + b$, $a\in \{0,\ldots \alpha-1\}$ and $b \in \{1,\ldots k\}$, of $A$ to be $k^{a}$ stacks of $v_b^{\alpha-a}$.

To get a better feel for the construction, we have given two example inputs in \cref{fig1}.
\begin{figure*}[h]
\begin{center}
$$ 
\begin{bmatrix}
\frac{1}{2} & -\frac{1}{2} & \frac{1}{2} & -\frac{1}{2} & \frac{1}{2} & -\frac{1}{2}  \\
-\frac{1}{2} & \frac{1}{2} & \frac{1}{2} & -\frac{1}{2} & \frac{1}{2} & -\frac{1}{2} \\
\frac{1}{2} & -\frac{1}{2} & -\frac{1}{2} & \frac{1}{2} & \frac{1}{2} & -\frac{1}{2}  \\
-\frac{1}{2} & \frac{1}{2} & -\frac{1}{2} & \frac{1}{2} & \frac{1}{2} & -\frac{1}{2} \\
\frac{1}{2} & -\frac{1}{2} & \frac{1}{2} & -\frac{1}{2} & -\frac{1}{2} & \frac{1}{2}  \\
-\frac{1}{2} & \frac{1}{2} & \frac{1}{2} & -\frac{1}{2} & -\frac{1}{2} & \frac{1}{2}  \\
\frac{1}{2} & -\frac{1}{2} & -\frac{1}{2} & \frac{1}{2} &-\frac{1}{2} & \frac{1}{2} \\
-\frac{1}{2} & \frac{1}{2} & -\frac{1}{2} & \frac{1}{2}  & -\frac{1}{2} & \frac{1}{2} \\
\end{bmatrix} ~~~~~~~~
\begin{bmatrix}
\frac{2}{3} & -\frac{1}{3} & -\frac{1}{3} & \frac{2}{3} & -\frac{1}{3} & -\frac{1}{3}\\
\frac{2}{3} & -\frac{1}{3} & -\frac{1}{3} & -\frac{1}{3} & \frac{2}{3} & -\frac{1}{3}\\
\frac{2}{3} & -\frac{1}{3} & -\frac{1}{3} & -\frac{1}{3} & -\frac{1}{3} & \frac{2}{3}\\
 -\frac{1}{3} & \frac{2}{3} & -\frac{1}{3} & \frac{2}{3} & -\frac{1}{3} & -\frac{1}{3}\\
-\frac{1}{3} & \frac{2}{3} & -\frac{1}{3} & -\frac{1}{3} & \frac{2}{3} & -\frac{1}{3}\\
-\frac{1}{3} & \frac{2}{3} &  -\frac{1}{3} & -\frac{1}{3} & -\frac{1}{3} & \frac{2}{3}\\
-\frac{1}{3} & -\frac{1}{3} & \frac{2}{3} & \frac{2}{3} & -\frac{1}{3} & -\frac{1}{3}\\
-\frac{1}{3} & -\frac{1}{3} & \frac{2}{3} & -\frac{1}{3} & \frac{2}{3} & -\frac{1}{3}\\
-\frac{1}{3} & -\frac{1}{3} & \frac{2}{3} & -\frac{1}{3} & -\frac{1}{3} & \frac{2}{3}\\
\end{bmatrix} 
$$
\end{center}
\label{fig1}
\caption{Benchmark construction for $k=2$ and $\alpha=3$ (left) and $k=3$ and $\alpha=2$ (right).}
\end{figure*}
\chris{TODO: Omar could you please rearrange the matrix on the right side in the figure such that it is consistent with the notation? We are generating them left to right, this matrix is written right to left.}

We now summarize the key properties of the benchmark.
To this end, we require a few notions.
Let $A$ be the input matrix. We slightly abuse notation an refer to $A_i$ as both the $i$th point as well as the $i$th row of the matrix $A$.
For a clustering $\mathcal{C}=\{C_1,\ldots ,C_k\}$, we define that the $n\times k$ indicator matrix $\tilde X$ induced by $\mathcal{C}$ via
$$ \tilde X_{i,j} = \begin{cases}1 & \text{if } A_i\in C_j \\
0 & \text{else.} \end{cases}$$
Furthermore, we will also use the $n\times k$ normalized clustering matrix $ X$ defined as
$$ \tilde X_{i,j} = \begin{cases}\frac{1}{\sqrt{|C_i|}} & \text{if } A_i\in C_j \\
0 & \text{else.} \end{cases}$$
We also recall the following lemma which will allow us to express the $k$-means cost of a clustering $\mathcal{C}$ with optimally chosen centers in terms of the cost of $X$ and $A$.
\begin{lemma}[Folklore]
Let $A$ be an arbitrary set of points and let $\mu(A) = \frac{1}{|A|}\sum_{p\in A} p$ be the mean. Then for any point $c$
$$ \sum_{p\in A} \|p-c\|^2 = \sum_{p\in A} \|p-\mu(A)\|^2 + |A|\cdot \|\mu(A)-c\|^2.$$
\end{lemma}
This lemma proves that for any given cluster $C_j$, the mean is the optimal choice of center. 
We also note that any two distinct columns of $X$ are orthogonal. Furthermore $\mathbf{1}\mathbf{1}^TA$ copies the mean into every entry of $A$. Combining these two observations, we see that the matrix $XX^TA$ maps the $i$th row of $A$ onto the mean of the cluster it is assigned to. Finally, define the Frobenius norm of an $n\times d$ $A$ by $\|A\|_F = \sqrt{\sum_{i=1}^n\sum_{j=1}^d A_{i,j}^2}$. Then the $k$-means cost of the clustering $\mathcal{C}$ is precisely
$$\|A-XX^TA\|_F^2.$$

Finally, we require the following distance measure on clusterings as proposed by Meila~\cite{Meila05,Meila06}. Given two clusterings $\mathcal{C}$ and $\mathcal{C'}$, the $k\times k$ confusion matrix $M$ is defined as
$$ M_{i,j} = |C_i\cap C'_j|.$$
Furthermore for the indicator matrices $\tilde X$ and $\tilde X'$ induced by $\mathcal{C}$ and $\mathcal{C'}$ we have the identity $M=\tilde X^T {\tilde X'}$.
Denote by $\Pi_k$ the set of all permutations over $k$ elements. Then the distance between  $\mathcal{C}$ and $\mathcal{C'}$ is defined as
$$d(\mathcal{C},\mathcal{C'}) = 1-\frac{1}{n}\underset{\max}{\pi\in \Pi_k} \sum_{i=1}^k M_{i,\pi(i)}.$$
Observe that for clusters that are identical, their distance is $0$. The maximum distance between any two $k$ clusterings is always $\frac{k-1}{k}$.


We are now ready to state the desired properties of our benchmark. The benchmark was designed to generate many clusterings such that
\begin{enumerate}
\item The distance between these clustering is maximized.
\item The clusterings have equal cost.
\item The clusterings are induced by a set of centers in $\mathbb{R}^d$.
\end{enumerate}

The first and second property ensure that (equally good) solutions we use for evaluation are spread out through the solution space. In other words, we are less likely to only focus on a set of solutions $\mathcal{S}$ for which a low distortion on one $S\in\mathcal{S}$ implies a low distortion for all elements of $\mathcal{S}$.
The third property is important as these are the only clusterings the (standard) coreset guarantee has to apply to. 


The solutions we consider as given as follows. For the columns $a\cdot k+1,\ldots (a+1)\cdot k$, we define the clustering $\mathcal{C}^{a} = \{C_1^a,\ldots C_k^a\}$ with 
$A_i\in C_j^a$ if and only if $A_{i,j} > 0$. Let $\tilde X^a$ and $X^{a}$ denote the indicator matrix and clustering matrix, respectively, as induced by $\mathcal{C}^{a}$.


\begin{fact}
For $a\neq a'$, we have $d(\mathcal{C}^{a},\mathcal{C}^{a'}) = 1-1/k$.
\end{fact}
\begin{proof}
Consider an arbitrary vector $v_i^{\ell}$. By construction, the positive entries of $v_i^{\ell}$ range from $k^{\ell-1}\cdot i+1$ to $k^{\ell-1}\cdot (i+1)$. Similarly, the positive entries for the vector $v_j^{\ell-1}$ range from range from $k^{\ell-2}\cdot j+1$ to $k^{\ell-2}\cdot (j+1)$. Therefore, concatenating $v_j^{\ell-1}$ $k$ times into a vector $v'$, $v'$ and $v_i^{\ell}$ can share at most one positive coordinate. Inductively, the same holds true for any concatenation of vectors $v_j^{\ell-h}$.
Thus, the two clusters induced by the columns formed by concatenating the vectors $v$ can share only a $1/k$ fraction of the points. Since each cluster consists of exactly $k^{\alpha}/k$ = $k^{\alpha-1}$ points, the confusion matrix $M$ only has entries $\frac{n}{k^2}$ and for any permutation $\pi$, we have $d(\mathcal{C}^{a},\mathcal{C}^{a'}) = 1-1/k$.
\end{proof}

\begin{fact}
For all $a,a'\in \{0,\ldots \alpha-1\}$, we have 
$$\|A-X^{a}(X^{a})^TA\|_F^2=(\alpha-1)\cdot (k-1)\cdot k^{\alpha-1}.$$
\end{fact}
\begin{proof}
Without loss of generality, assume $a=0$; the proof otherwise holds by rearranging rows and columns of $A$ due to the preceding fact. We first note that for any point $A_i \in C_i$, the coordinates $A_{i,\ell}$ are identical for $\ell <k$. Furthermore for the column $\ell\geq k$, we have by construction $\sum_{A_i\in C_j} A_{i,\ell} = k^{\alpha-1}\cdot \frac{k-1}{k} + (k^{\alpha}-k^{\alpha-1})\frac{1}{k}=k^{\alpha-1}\cdot (\frac{k-1}{k} - (k-1)\frac{1}{k}) = 0.$ Therefore, the mean of $C_j$ satisfies $\mu(C_j)_{\ell} = \begin{cases}A_{i,\ell} &\text{if }\ell<k \\
0 &\text{else.}\end{cases}$. Thus, $(A-X^{a}(X^{a})^TA)_{i,\ell} =\begin{cases}A_{i,\ell}&\text{if }\ell\geq k\\ 0 & \text{if }\ell<k\end{cases}$ and the cost is $(\alpha-1)\cdot k\cdot \left( k^{\alpha-1} \cdot \left(\frac{k-1}{k}\right)^2 + k^{\alpha}-k^{\alpha-1}\cdot \left(\frac{1}{k}\right)^2\right) = (\alpha-1)(k-1)k^{\alpha-1}.$
\end{proof}

Finally, we show that the means for the clustering $\mathcal{C}^{a}$ also induce $\mathcal{C}^{a}$.

\begin{fact}
For a clustering $\mathcal{C}^{a}$, let $\mu(C_j^{a})$ denote the mean of cluster $C_j^a$. Then every point of is assigned to its closest center.
\end{fact}
\begin{proof}
Again, we assume without loss of generality $a=0$.
Let $A_i$ be an arbitrary point of cluster $C_{h}^{a}$ and consider the mean $\mu(C_j^a)_{\ell} = \begin{cases}A_{i,\ell} &\text{if }\ell<k \\
0 &\text{else.}\end{cases}$ of cluster $C_j^a$. By definition, the positive coordinates of $A_i$ are not equal to the positive coordinates of $\mu(C_j^a)$. The only difference in coordinates between the means of $\mu(C_j^a)$ and $\mu(C_h^a)$ are the first $k$ coordinates, as the rest are $0$.
But here the coordinates of $\mu(C_h^a)$ and $A_i$ are identical, hence $\mu(C_j^a)$ cannot be closer to $A_i$.
\end{proof}




% 


\section{Experiments} \label{sec:experiments}
In this section, we present how we evaluated different algorithms. First, we propose our evaluation procedure which gauges the quality of coresets. Then, we describe the data sets used for the empirical evaluation and our experimental setup. Finally, we detail the outcome of the experiments and our interpretation of the results.

\subsection{Evaluation Procedure}
Accurately evaluating a $k$-means coreset of a real-world data set requires constructing a solution $\calS$ (a set of $k$ centers) which results in a maximal distortion. Finding such a solution, however, is difficult. Instead, we can estimate the quality of a given coreset by finding meaningful candidate solutions. 

One approach is to use $k$-means++ as follows: compute a coreset $\Omega$ on a real-world data set $A$. Then, run $k$-means++ on $\Omega$ to find a set of $k$ centers. Repeat the $k$-means++ algorithm 5 times and pick $\calS$ to be the set of centers with the largest distortion. The main advantage of this approach is that $k$-means++ can uncover natural cluster structures in the data.

Another approach is to randomly generate a candidate solution $\calS$. For example, one could generate $k$ random points inside the minimum enclosing ball (MEB) of a coreset $\Omega$. This can be repeated 5 times. Then, a  candidate solution is the set of points with the largest distortion. While it is very fast to generate candidate solutions in this manner, this method has its drawbacks. It is not readily apparent how to define a distribution of meaningful solutions from which to sample. Moreover, a randomly drawn solution, which does not exploit the behavior of a coreset construction, is less likely to yield a worst-case candidate solution. Nevertheless, we apply both $k$-means and random sampling inside the MEB to generate candidate solutions in our evaluations.

Granted the usefulness of evaluating coresets on real-world data sets, it can be tricky to gauge the general performance of coreset algorithms using only a small selection of data sets. For this reason, we used our benchmark to complement the evaluation on real-world data sets. The benchmark accomplishes two important tasks. First, the benchmark allows us to quickly find a bad solution because both good and bad clusterings are known a priori. It is unclear how to find bad clusterings for real-world data sets. Second, it is easier to make a fair comparison of different coreset constructions because the benchmark is known to generate hard instances for all known coreset algorithms. This cannot be said for real-world data sets. For the benchmark, we computed the distortion following the evaluation procedure described in~\cref{sec:benchmark}. 


Every randomized coreset construction was repeated $10$ times. We aggregated the reported distortions by taking the average over all $10$ evaluations. 
In addition, we preprocessed the data using PCA (compare Section~\ref{sec:dim_reduction}).


% We now present the empirical evaluation of these coresets.
% We ran two kinds of experiments. On real-world data sets, we merely computed a coreset $\Omega$, followed by running $k$-means++ on $\Omega$. 
% The $k$-means++ algorithm was repeated 5 times, each yielding a solution $\calS_i$, and as the best lower bound on the distortion we used the largest ratio $\max_i\left(\max\left(\frac{\cost_A(\calS_i)}{\cost_{\Omega}(\calS_i)},\frac{\cost_{\Omega}(\calS_i)}{\cost_A(\calS_i)}\right)\right)$.
% For the benchmark, we used the evaluation as proposed in Section~\ref{sec:benchmark}. In addition, we also determined the distortion via simply running the $k$-means++ algorithm. 

% Except for BICO, which is deterministic, this experiment was repeated for each coreset algorithm $10$ times. \omar{Experiments were repeated 10 times including BICO. Although vanilla BICO is deterministic, we used BICO with heuristic speed optimizations which are stochastic.} We aggregated the reported distortions by taking the maximum over all $10$ evaluations. In addition, we also preprocessed the data using the dimension reduction techniques described in Section~\ref{sec:algorithms}.


\subsection{Data sets}
We conducted experiments on five real-world data sets and four instances of our benchmark. Benchmark instances were generated to match approximately the sizes of the real-world data sets. The sizes of the data sets are summarized in ~\cref{tab:real-world-datasets-overview}. 
% We generated a set of instances with no scaling i.e., $\beta=1.0$ (referred to as \textit{Benchmark-1.0}) and with maximum scaling; $\beta = 2.0$ (\textit{Benchmark-2.0}).
We now provide a brief description of each of the real-world data sets.

The \textit{Census}\footnote{\url{https://archive.ics.uci.edu/ml/datasets/US+Census+Data+(1990)}} dataset is a small subset of the Public Use Microdata Samples from 1990 US census. It consists of demographic information encoded as 68 categorical attributes of 2,458,285 individuals. 

\textit{Covertype}\footnote{\url{https://archive.ics.uci.edu/ml/datasets/covertype}} is comprised of cartographic descriptions and forest cover type of four wilderness areas in the Roosevelt National Forest of Northern Colorado in the US. It consists of 581,012 records, 54 cartographic variables and one class variable. Although \textit{Covertype} was originally made for classification tasks, it is often used for clustering tasks by removing the class variable~\cite{AckermannMRSLS12}.

The data set with the fewest number of dimensions is \textit{Tower}\footnote{\url{http://homepages.uni-paderborn.de/frahling/coremeans.html}}. This data set consists of 4,915,200 rows and 3 features as it is a 2,560 by 1,920 picture of a tower on a hill where each pixel is represented by a RGB color value. 

% We used the datasets \textit{Census}, \textit{Covertype} and \textit{Tower} as these are often used to evaluate the performance of coreset algorithms. 

% To include larger datasets in evaluation, we used \textit{Caltech} and \textit{NYTimes}. 

Inspired by~\cite{FGSSS13}, \textit{Caltech} was created by computing SIFT features from the images in the Caltech101\footnote{\url{http://www.vision.caltech.edu/Image_Datasets/Caltech101/}} image database. This database contains pictures of objects partitioned into 101 categories. Disregarding the categories, we concatenated the 128-dimensional SIFT vectors from each image into one large data matrix with 3,680,458 rows and 128 columns. 

% \textit{NYTimes}\footnote{\url{https://archive.ics.uci.edu/ml/datasets/bag+of+words}} dataset in our experiments as the number of dimensions is very large.
\textit{NYTimes}\footnote{\url{https://archive.ics.uci.edu/ml/datasets/bag+of+words}} is a dataset composed of the bag-of-words (BOW) representations of 300,000 news articles from The New York Times. The vocabulary size of the text collection is 102,660. Due to the BOW encoding, \textit{NYTimes} has a very large number of dimensions and is highly sparse. To make processing feasible, we reduced the number of dimensions to 100 using terminal embeddings.

\subsection{Preprocessing \& Experimental Setup}
To understand how denoising effects the quality of the outputted coresets, we applied Principal Component Analysis (PCA) on \textit{Caltech}, \textit{Census} and \textit{Covertype} by using the $k$ singular vectors corresponding to the largest singular values. For these three data sets, we preserved the dimensions of the original data.  
The \textit{NYTimes} dataset did not permit the preservation of dimensions as the number of dimensions is very large. In this case, we used PCA to reduce the dimensions to $k$.
We did not perform any preprocessing on \textit{Tower} due to its low dimensionality.

We followed the same experimental procedure with respect to the choice of parameter values for the algorithms as prior works~\cite{AckermannMRSLS12, FGSSS13}. For the target coreset size, we used $200k$ for all our experiments. On \textit{Caltech}, \textit{Census},  \textit{Covertype} and \textit{NYTimes}, we used $k$ values in $\{10, 20, 30, 40, 50\}$, while for \textit{Tower} we used larger cluster sizes $k \in \{20, 40, 60, 80, 100\}$. On the benchmark instances, we settled on $k \in \{10, 20, 30, 40\}$ as a reasonable trade-off between running time and data set size.


We implemented Sensitivity Sampling, Group Sampling, Ray Maker, and StreamKM++ in C++. The source code can be found on GitHub\footnote{Link to repository will be provided later.}. For BICO, we used the authors' reference implementation\footnote{\url{https://ls2-www.cs.tu-dortmund.de/grav/en/bico}}. The source code was compiled with gcc 9.3.0. The experiments were performed on a machine with 14 cores (3.3 GHz) and 256 GB of memory.


%
\begin{center}
\begin{table}
	
\begin{minipage}{0.5\textwidth}
	%\begin{center}%\centering
% 	\resizebox{\textwidth}{!}{
	\begin{tabular}{lrr}
		\toprule
        
		    & Data points
		    & Dimensions
            \\
		\midrule
		\textit{Caltech}
    		& 3,680,458
    		& 128
    		\\
		\textit{Census}
    		& 2,458,285
    		& 68
    		\\
	    \textit{Covertype}
    	    & 581,012
    		& 54
    		\\
	    \textit{NYTimes}
    	    & 500,000
    		& 102,660
    		\\
        \textit{Tower}
            & 4,915,200
    		& 3
    		\\
		\bottomrule
	\end{tabular}\\
%	\end{center}
% 	}
\end{minipage}
\begin{minipage}{0.5\textwidth}
%	\begin{center}%\centering
% 	\resizebox{\textwidth}{!}{
	\begin{tabular}{rrrr}
		\toprule
        $k$
		    & $\alpha$
		    & Data points
		    & Dimensions
            \\
		\midrule
        10
    		& 6
    		& 1,000,000
    		& 60
    		\\
        20
    		& 5
    		& 3,200,000
    		& 100
    		\\
        30
    		& 4
    		& 810,000
    		& 120
    		\\
        40
    		& 4
    		& 2,560,000
    		& 160
    		\\
    %     50
    % 		& 4
    % 		& 6,250,000
    % 		& 200
    % 		\\
		\bottomrule
	\end{tabular}\\
%	\end{center}
% 	}
\end{minipage}
\caption{Size of real world data sets (left) and evaluated benchmarks (right).}
\label{tab:real-world-datasets-overview}
\end{table}
\end{center}


%



\begin{figure*}
  \includegraphics[width=.69\linewidth]{figures/distortions-mean-Tower.pdf}
  \newline \newline
  \subfloat{
    \includegraphics[width=0.5\textwidth]{figures/distortions-mean-Covertype.pdf}
  }
  \subfloat{
    \includegraphics[width=.5\linewidth]{figures/distortions-mean-Covertype+PCA.pdf}
  }
  \newline \newline
  \subfloat{
    \includegraphics[width=0.5\textwidth]{figures/distortions-mean-Census.pdf}
  }
  \subfloat{
    \includegraphics[width=.5\linewidth]{figures/distortions-mean-Census+PCA.pdf}
  }
  \newline \newline
  \subfloat{
    \includegraphics[width=0.5\textwidth]{figures/distortions-mean-Caltech.pdf}
  }
  \subfloat{
    \includegraphics[width=.5\linewidth]{figures/distortions-mean-Caltech+PCA.pdf}
  }
  \newline \newline
  \subfloat{
    \includegraphics[width=0.5\textwidth]{figures/distortions-mean-NYTimes.pdf}
  }
  \subfloat{
    \includegraphics[width=.5\linewidth]{figures/distortions-mean-NYTimes+PCA.pdf}
  }
  \newline \newline
%   \subfloat{
%     \\[1ex]
%     \includegraphics[width=0.5\textwidth]{figures/distortions-mean-Benchmark.pdf}
%   }
  \subfloat{
    \includegraphics[width=0.165\textwidth]{figures/distortions-mean-Benchmark-k10.pdf}
    \includegraphics[width=0.165\textwidth]{figures/distortions-mean-Benchmark-k20.pdf}
    \includegraphics[width=0.165\textwidth]{figures/distortions-mean-Benchmark-k30.pdf}
    \includegraphics[width=0.331\textwidth]{figures/distortions-mean-Benchmark-k40.pdf}
  }
%   \subfloat{
%     \includegraphics[width=.5\linewidth]{figures/boxplot-Benchmark-GS-SS-StreamKM.pdf}
%   }
  \caption{The distortions of the evaluated coreset algorithms on five real-world data sets and on four benchmark instances. The axis is non-linear as otherwise the bars for Sensitivity Sampling and Group Sampling would disappear on the plots. Their distortions are very close to 1.}
  \label{fig:distortions}
\end{figure*}


%\begin{figure*}
%  \caption{The average costs of running the evaluated coreset algorithms multiple times on different data sets. In general, the five coreset algorithms are able to compute coresets which result in solutions with comparable costs on the different real-world data sets. The differences in cost is more noticeable on the benchmark instances. Here, Senstivity Sampling is the winner because it seems to be better at capturing the correct ``clusters'' inherent in the benchmark instances.}
%  \label{fig:real-costs}
%  \includegraphics[width=.67\linewidth]{figures/real-costs-Tower.pdf}
%  \newline
%  \subfloat{
%    \includegraphics[width=0.5\textwidth]{figures/real-costs-Covertype.pdf}
%  }
%  \subfloat{
%    \includegraphics[width=.5\linewidth]{figures/real-costs-Covertype+PCA.pdf}
%  }
%  \newline\newline
%  \subfloat{
%    \includegraphics[width=0.5\textwidth]{figures/real-costs-Census.pdf}
%  }
%  \subfloat{
%    \includegraphics[width=.5\linewidth]{figures/real-costs-Census+PCA.pdf}
%  }
%  \newline\newline
%  \subfloat{
%    \includegraphics[width=0.5\textwidth]{figures/real-costs-Caltech.pdf}
%  }
%  \subfloat{
%    \includegraphics[width=.5\linewidth]{figures/real-costs-Caltech+PCA.pdf}
%  }
%  \newline\newline
%  \subfloat{
%    \includegraphics[width=0.5\textwidth]{figures/real-costs-NYTimes.pdf}
%  }
%  \subfloat{
%    \includegraphics[width=.5\linewidth]{figures/real-costs-NYTimes+PCA.pdf}
%  }
%  \newline\newline
%  \subfloat{
%    \includegraphics[width=0.15\textwidth]{figures/real-costs-Benchmark-k10.pdf}
%    \includegraphics[width=0.165\textwidth]{figures/real-costs-Benchmark-k20.pdf}
%    \includegraphics[width=0.16\textwidth]{figures/real-costs-Benchmark-k30.pdf}
%    \includegraphics[width=0.31\textwidth]{figures/real-costs-Benchmark-k40.pdf}
%  }
%\end{figure*}





\subsection{Outcome of Experiments}
We summarized the distortions in \cref{fig:distortions}, for numerical variance we refer to \cref{tab:distortions-mean-std} in the appendix.
All five algorithms are matched on the \textit{Tower} dataset. The worst distortions across the algorithms are close to 1, and performance between the algorithms is negligible. The performance difference between sampling-based and movement-based methods become more pronounced as the number of dimensions increase. On \textit{Covertype} with its 54 features, Ray Maker performs the worst followed by BICO and Group Sampling while Sensitivity Sampling and StreamKM++ perform the best. Differences in performance are more noticeable on \textit{Census}, \textit{Caltech}, and \textit{NYTimes}  where methods based on importance sampling perform much better. Sensitivity Sampling and Group Sampling perform the best, StreamKM++ come in second while BICO and Ray Maker perform the worst across these data sets.
On the \textit{Benchmark}, Ray Maker is the worst while Sensitivty Sampling and Group Sampling are the best. StreamKM++ performs also very well compared to BICO.

Reducing noise with PCA helped boost the performance on real-world data sets with large number of dimensions. On \textit{Covertype}, PCA does not change the performance numbers by much. On \textit{Census} ($d=68$), the preprocessing step with PCA improves the performance of BICO and Ray Maker slightly for lower values of $k$. The distortions of BICO and Ray Maker are reduced markedly on \textit{Caltech} ($d=128$) after applying PCA. 


\subsection{Interpretation of Experimental Results}



\subsubsection*{Optimization versus Compression}
While all five algorithms are equally matched when optimizing on the candidate coresets, coreset quality performance differ significantly (see \cref{fig:distortions}). We omit tables and plots detailing the cost due to space restrictions. For all data sets, the obtained costs differed insignificantly for all values of $k$.

The quality of the coreset itself can be closely tied to the change in cost with increasing number of centers. It is not uncommon for the $k$-means cost of real-world data sets to drop significantly for larger values of $k$.
~\cref{fig:cost-curves-real-world-datasets} illustrates this behavior for several real-world data sets. The more the curve bends, the less of a difference there is between computing a coreset and a clustering with low cost. For data sets with a L-shaped cost curve, a coreset algorithm adding more centers to the coreset will seem to be performing well when evaluating it based on the outcome of the optimization.
\textit{Tower} is a good example of a data set where optimization is very close to compression. Its cost curve bends the most which indicates that adding more centers help reduce the cost. One of the strengths of the benchmark is that there is no way of reducing the cost without capturing the right subclusters within a benchmark instance. This means that the cost does not decrease markedly beyond a certain value of $k$ even if more centers are added, see~\cref{fig:cost-curves-benchmark} in the appendix. 

For BICO, Ray Maker, and StreamKM++, there is a correlation between the steepness of the cost curve for a data set and the distortion of the generated coreset. 
On data sets where the curve is less steep, we observed higher distortions. The effect is more pronounced for the movement-based algorithms (BICO and Ray Maker) than for StreamKM++. The other two sampling-based approaches (Group Sampling and Sensitivity Sampling) seem to be free from this behavior as they consistently generate high quality coresets irrespective of the shape of cost curve.







\begin{figure}
  \includegraphics[width=1\linewidth]{figures/cost-curves-real-world-datasets.pdf}
  \caption{Depicts how clustering costs of five real-world data sets decrease as the number of centers increase. 
  The most widely used data sets for evaluating coresets are \textit{Tower}, \textit{Covertype}, and \textit{Census}, while \textit{Caltech} is rarely used and \textit{NYTimes} has never been used before.
  Plotting the cost curve allows us to study whether we can observe a difference between coreset construction and optimization in a data set when evaluating a coreset based on cost.
  }
  \label{fig:cost-curves-real-world-datasets}
\end{figure}



\subsubsection*{Movement-based versus Sampling-based Approaches}

In general, movement-based constructions perform the worst in terms of coreset quality. 
We observed that BICO and Ray Maker have the highest distortions across all data sets including on the benchmark instances. Among the sampling-based algorithms, Sensitive Sampling performs well with Group Sampling generally being competitive. This runs contrary to theory where Group Sampling has the better (currently known) theoretical bounds. StreamKM++ is an interesting case. Like the movement-based methods, its distortion increases with the dimension. Nevertheless, it generally performs significantly better than BICO and Ray Maker. This can be attributed to the fact that the coreset produced by StreamKM++ consists entirely of $k$-means++ centers weighted by the number of points of a minimal cost assignment. This is similar to movement-based algorithms such as BICO. Nevertheless, it also retains some of the performance from pure importance schemes.

In practice as well as in theory, the distortion of movement-based algorithms are affected by the dimension. By comparison, sampling-based algorithms are affected very little. Theoretically, there should not exist a difference, as the sampling bounds are independent of the dimension. What little effect can be observed is likely due to PCA making it easier to find low cost solutions that form the backbone of all coreset constructions. StreamKM++ is an interesting case, as it is still affected by the dimension, though less than the other movement based methods, due its performance without dimension reduction being significantly better than the worst-case bounds would suggest. 





\subsubsection*{Impact of PCA}

On almost all our data sets, the performance improves when input data is preprocessed with PCA, especially for the movement-based algorithms. Empirically, the more noise is removed (i.e., small $k$ value), the lower the distortion. Notice that $k$ is the number of principal components that the input data is projected on to. The rest of the low variance components are treated as noise and removed. Method utilizing sampling (Group Sampling, Sensitivity Sampling and StreamKM++) are less effected by the preprocessing step. On \textit{Covertype}, PCA does not change the distortions by much because almost all the variance in the data is explained by the first five principal components (see~\cref{fig:explained-variance-pca}). 
On \textit{Caltech} and \textit{NYTimes}, the quality of the coresets by BICO and Ray Maker improves greatly because the noise removal is more aggressive. Even if the quality is much better for movement-based coreset constructions due to PCA, importance sampling methods are still superior when it comes to the quality of the compression. Summarizing, all methods benefit from PCA, and in case of movement based constructions, we consider PCA a necessary preprocessing step. For the sampling based methods, the computational expense of using PCA in preprocessing does not seem justify the comparatively meagre gains in coreset distortion.




% \section{Conclusion} \label{sec:conclusion}
In this work, we studied how to assess the quality of $k$-means coresets computed by state-of-the-art algorithms. It is generally hard to measure the quality of a coreset because it requires computing the worst-case distortion. Due to this difficulty, earlier works evaluated coresets by the outcome of an optimization algorithm. This method of comparison has the drawback that it is more likely to measure the performance of the underlying optimization problem, rather than evaluating coresets. As a alternative, we proposed a new evaluation procedure which can estimate the quality of coresets on real-world data sets. To complement this, we also proposed a benchmark framework which provably generates hard instances for all known $k$-means coreset algorithms. We evaluated the quality of five $k$-means coreset algorithms on five real-world datasets and four instances of the proposed benchmark. We experimented with both movement-based and sampling-based coreset algorithms. We found that while all algorithms produce coresets which yield similar low cost clusterings, sampling-based methods are superior to movement-based algorithms. 
\chris{Add the k-means++ question here. I'm not sure if I ever mentioned that before. If you came up with it by yourself, I am very impressed. It also flowed naturally in what you were writing.}
\omar{I wish I came up with it myself! I do not yet understand the proofs. Anyway, I moved the section ``BICO versus StreamKM++'' to here and tried to make it shorter. What do you think?}

Surprisingly, the quality of the coresets produced by StreamKM++ is significantly better than what one would expect from its theoretical analysis. StreamKM++ is derived from a movement-based construction similar to BICO. Emprically, StreamKM++ is notably better than BICO across all data sets, and especially on high dimensional data. This begs the question whether there exist a better theoretical analysis for StreamKM++. We leave this as an open problem for future research.



% \subsubsection*{Acknowledgments}
% Use unnumbered third level headings for the acknowledgments. All
% acknowledgments, including those to funding agencies, go at the end of the paper.


\bibliography{references}
\bibliographystyle{iclr2022_conference}

% \appendix
% \section{Appendix}
% You may include other additional sections here.

\end{document}
